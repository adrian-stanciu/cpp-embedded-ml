% vim: set tw=78 aw sw=2 sts=2 noet:

\section{Motivation}

\begin{frame}{Machine Learning (ML)}
\begin{itemize}
  \item Subfield of Artificial Inteligence (AI)
  \item Enables computers to learn from data and then use that knowledge to
  make predictions
  \item Applications:
	\begin{itemize}
	  \item Computer vision
	  \item Information retrieval
	  \item Medicine
	  \item Recommender systems
	  \item Search engines
	\end{itemize}
\end{itemize}
\end{frame}

\begin{frame}{Embedded devices}
\begin{itemize}
  \item Computing devices designed to perform specific tasks within larger systems
 \item Applications:
	\begin{itemize}
	  \item Consumer electronics (smart TVs, mobile phones)
	  \item Home automation (thermostats, lighting control systems)
	  \item Medical equipment (pacemakers, insulin pumps)
	\end{itemize}
  \item Characteristics:
	\begin{itemize}
	  \item Limited computer hardware resources
	  \item Low power consumption
	  \item May have real-time performance constraints
	\end{itemize}
\end{itemize}
\end{frame}

\begin{frame}{Machine learning on embedded devices}
\begin{itemize}
 \item Alternative to cloud-based machine learning
 \item Advantages:
	\begin{itemize}
	  \item Real-time processing
	  \item Low latency
	  \item Reduced bandwidth usage
	  \item Offline operation
	  \item Improved privacy
	\end{itemize}
  \item Disadvantages:
	\begin{itemize}
	  \item Compatibility with various hardware and software platforms
	  \item Maintenance and updates
	\end{itemize}
\end{itemize}
\end{frame}

\begin{frame}{Using C++ for machine learning on embedded devices}
C++ is widely used in embedded systems:
\begin{itemize}
 \item Was designed with efficiency in mind
 \item Offers low-level access to hardware resources
 \item Provides high-level abstractions
 \item Is supported on most hardware and software platforms
\end{itemize}
\end{frame}

