% vim: set tw=78 aw sw=2 sts=2 noet:

\section{Hand gesture recognition}

\begin{frame}{Rock-Paper-Scissors}
  \begin{figure}
	\includegraphics[width=\linewidth,height=0.75\textheight,keepaspectratio]{images/502px-Rock-paper-scissors.svg.png}
	\caption{wikipedia.org}
  \end{figure}
\end{frame}

\begin{frame}{Data}
  \begin{itemize}
	\item Google MediaPipe's Rock-Paper-Scissors dataset for hand gesture
	recognition
	  \begin{itemize}
		\item Contains 125 images for each class
		\item Images have various sizes
		\item Images have 3 color channels per pixel (RGB)
	  \end{itemize}
	\item Laurence Moroney's Rock-Paper-Scissors dataset (published by Sani
	Kamal on Kaggle)
	  \begin{itemize}
		\item Contains 964 images for each class split into training data
		(840) and testing data (124)
		\item Images are 300x300 pixels
		\item Images have 3 color channels per pixel (RGB)
	  \end{itemize}
  \end{itemize}
\end{frame}

\begin{frame}{Train a Rock-Paper-Scissors model}
  \begin{itemize}
	\item Transfer learning from a ResNet50 model
	\item 80/20 split between training and validation data
	\item Run in Python, on a laptop
  \end{itemize}
  \begin{table}
	\begin{tabular}{|c|c|c|}
	  \hline
		\textbf{Dataset} & \textbf{Train duration} & \textbf{Model size (MB)} \\
	  \hline
		Small & 2m15s & 23 \\
	  \hline
		Big & 7m10s & 23 \\
	  \hline
	\end{tabular}
  \end{table}
\end{frame}

\begin{frame}{Test the Rock-Paper-Scissors model}
  \begin{itemize}
	\item Using the testing data of the big dataset (124 images per class)
	\item Small dataset results:
  \begin{table}
	\begin{tabular}{|c|c|c|}
	  \hline
		\textbf{Gesture} & \textbf{Correct predictions} & \textbf{Accuracy (\%)} \\
	  \hline
		Rock & 108 & 87.09 \\
	  \hline
		Paper & 124 & 100 \\
	  \hline
		Scissors & 8 & 6.45 \\
	  \hline
	  \hline
		All & 240 out of 372 & 64.51 \\
	  \hline
	\end{tabular}
  \end{table}
	\item Big dataset results:
  \begin{table}
	\begin{tabular}{|c|c|c|}
	  \hline
		\textbf{Gesture} & \textbf{Correct predictions} & \textbf{Accuracy (\%)} \\
	  \hline
		Rock & 124 & 100 \\
	  \hline
		Paper & 122 & 98.38 \\
	  \hline
		Scissors & 97 & 78.22 \\
	  \hline
	  \hline
		All & 343 out of 372 & 92.20 \\
	  \hline
	\end{tabular}
  \end{table}
  \end{itemize}
\end{frame}

\begin{frame}{Performance}
  \begin{itemize}
	\item Compilation duration (on Raspberry Pi): 35s
	\item Binary size: 67KB
	\item Running duration: 1s
  \end{itemize}
  \begin{table}
    {\tiny
	\begin{tabular}{|c|c|}
	  \hline
		\textbf{Number of threads} & \textbf{Inference duration (ms)} \\
	  \hline
		1 & 151 \\
	  \hline
		2 & 98 \\
	  \hline
		3 & 80 \\
	  \hline
		4 & 71 \\
	  \hline
		5 & 189 \\
	  \hline
	\end{tabular}
	}
  \end{table}
  \begin{figure}
	\includegraphics[width=\linewidth,height=0.45\textheight,keepaspectratio]{images/inference_duration_rps.pdf}
  \end{figure}
  \begin{itemize}
	\item Memory consumption with 4 threads: 118MB
  \end{itemize}
\end{frame}

\begin{frame}{Capture camera stream}
  \begin{itemize}
	\item libcamera backend
	\item gstreamer middleware
	\item OpenCV frontend
  \end{itemize}
\end{frame}

\begin{frame}[fragile]{Capture camera stream}
  \lstset{basicstyle=\ttfamily\small, showstringspaces=false, numbers=left,
  columns=fullflexible}
  \begin{lstlisting}
static constexpr const char *GstreamerPipeline{R"(
    libcamerasrc !
    video/x-raw,
    width=(int)640,
    height=(int)480,
    framerate=(fraction)10/1 !
    videoconvert !
    appsink
)"};

cv::VideoCapture camera{
  GstreamerPipeline,
  cv::CAP_GSTREAMER
};

cv::Mat image;
camera.read(image);
  \end{lstlisting}
\end{frame}

\begin{frame}{Live inference}
  \begin{enumerate}
	\item Create image classifier
	\item Open camera
	\item Do in a loop:
	\begin{enumerate}
	  \item Read image from camera
	  \item Use classifier to run inference on image
	  \item Show predictions
	\end{enumerate}
  \end{enumerate}
\end{frame}

\begin{frame}{Rock-Paper-Scissors game}
On each round the AI:
  \begin{enumerate}
	\item Chooses Rock, Paper or Scissors uniformly at random
	\item Infers the human's hand gesture
	\item Computes the outcome
  \end{enumerate}
\end{frame}

